\documentclass[ITR, MA, english, intermediate]{LSR_thesis} 
\graphicspath{{pics/}}

%% Options:
%% LSR: LSR Template with Prof. Buss as default
%% ITR: ITR Template with Prof. Hirche as default

%% BA: Bachelorarbeit / Bachelor thesis
%% MA: Masterarbeit / Master thesis
%% HS: Hauptseminar / Scientific seminar
%% PP: Projektpraktikum / practical course
%% IP: Ingenieurpraxis
%% FP: Forschungspraxis
%% SeA: Semesterarbeit (MW)

%% english
%% german

%% final
%% intermediate

%% tutorial -> remove flag such that help files and todos are removed

%%________________________________________________%
%%_________CUSTOMIZE LATEX ONLY IN customize.tex!_% 
%%_________ DO NOT MODIFY THE TEMPLATE!___________%
%%________________________________________________%
%% add customize first so you can access your commands in gloss
\input{./include/customize.tex} % add custom commands etc in this file 
%% %%%%%%%%%%%%%%%%%%%%%%%%%%%%%%%%%%%%%%%%%%%%%%%%%%%%%%%%%%%%%%
% NOTE: Using this is optional.
% 	Nonetheless, feel free to include this file and adjust the 
% 	examples below as needed.
%   
%
% Further reading:
%	https://ctan.org/pkg/glossaries?lang=en
%%%%%%%%%%%%%%%%%%%%%%%%%%%%%%%%%%%%%%%%%%%%%%%%%%%%%%%%%%%%%%


% include packages
\usepackage[acronyms, style=alttree, toc=true, nonumberlist, nogroupskip, nopostdot, automake]{glossaries}

%%%%%%%%%%%%%%%%%%%%%%%%%%%%%%%%%%
% DEFINE HEADINGS AND CATEGORIES %
%%%%%%%%%%%%%%%%%%%%%%%%%%%%%%%%%%
\newcommand{\Notation}{Notation}
\newcommand{\Symbols}{List of Symbols}

\newglossary{notation}{not}{nt}{\Notation}
\newglossary{symbols}{sym}{sbl}{\Symbols}

\glssetwidest{THISWIDE} % adjust width of glossary

\makeglossaries

%===========%
% ACRONYMS	%
%===========%
\newacronym{MPC}{MPC}{model-predictive control}
\newacronym{BIBO}{BIBO}{bounded-input bounded-output}
\newacronym{HRC}{HRC}{human-robot collaboration}

%============%
% SYMBOLS	%
%===========%
\newglossaryentry{control}{type=symbols,
	sort={control},
	name={\ensuremath{\vec{u}}},
	description={control input vector}
}
\newglossaryentry{uk}{type=symbols,
	sort={control},
	name={\ensuremath{\vec{u}_k}},
	description={control input vector with time step}
}

\newglossaryentry{xk}{type=symbols,
	sort={state},
	name={\ensuremath{\vec{x}_k}},
	description={state vector with time step}
}

%============%
% NOTATION	%
%===========%
\newglossaryentry{vector}{type=notation,
	sort={vector},
	name={\ensuremath{\vec{x}_n}},
	description={$n$-dimensional vector named $x$}
}	

\newglossaryentry{matrix}{type=notation,
	sort={vector-matrix},
	name={\ensuremath{\Mat{x}_{m\times n}}},
	plural={matrices},
	user1={Mat},
	description={\ensuremath{m\times n} dimensional matrix  named \ensuremath{X}}
}

\glsaddall % Print all glossary entries

%%%% Add GLOSSARIES at end of thesis
\newcommand{\AddMyGloss}{
	\cleardoublepage
   	\printglossary[type=acronym, nogroupskip]
	
	\ifdefined\Notation
		\cleardoublepage
		\printglossary[type=notation, nogroupskip]
	\fi
	
  	\ifdefined\Symbols
  		\cleardoublepage
		\printglossary[type=symbols, nogroupskip]
	\fi
}

		% add your glossary in this file 

%\usepackage[bb=boondox]{mathalfa}

\usepackage{algorithm}
\usepackage{algpseudocode}

\usepackage{empheq}

\usepackage{tikz}
\usepackage{pgfplots}

\usepackage{array}
\usepackage{multirow}

\pgfplotsset{compat=1.18}
\usetikzlibrary{arrows,shapes,backgrounds,patterns,fadings,decorations,
	positioning,external,arrows.meta,pgfplots.fillbetween, pgfplots.groupplots}

%% start document
\begin{document}

%%%%%%%%%%%%%%%%%%%%%%%%%%%%%%%%%%%%%%%%%%%%%%%%%%%%%%%%%%%%%%%
%%%%%%%%%%%%%%%%%%% Title Page %%%%%%%%%%%%%%%%%%%%%%%%%%%%%%%%
%%%%%%%%%%%%%%%%%%%%%%%%%%%%%%%%%%%%%%%%%%%%%%%%%%%%%%%%%%%%%%%
%% English title:
\title{Kernel Embedding for Particle Gibbs-Based Optimal Control}

%% German title (for German theses only)
%\title{Die Antwort auf Alles und Mehr - Ein Trauerspiel in 4 Akten}
% and English translation (optional)
\titletranslation{
%Die Verwendung eines Deutschen Untertitels obliegt der Verantwortung der entsprechenden Betreuer
}

%% insert your data here:
\student{Lukas Hochschwarzer} % your name
\studtitle{cand. ing.} % your title
\emailaddress{l.hochschwarzer@gmail.com} % your email address, pls use @tum.de
 
%-if more students are involved (e.g., PP)-
%\studenttwo{Second Student}
%\studtitletwo{} 
%\studentthree{} 
%\studtitlethree{} 
%\studentfour{} 
%\studtitlefour{} 
%-----------------------------------------

\supervisor{M.Sc. Robert Lefringhausen} % your supervisor
\finalrep{06.11.2024} % submission date

\maketitle

%%%%%%%%%%%%%%%%%%%%%%%%%%%%%%%%%%%%%%%%%%%%%%%%%%%%%%%%%%%%%%%
%%%%%%%%%%%%%%%%%%% Second Page %%%%%%%%%%%%%%%%%%%%%%%%%%%%%%%
%%%%%%%%%%%%%%%%%%%%%%%%%%%%%%%%%%%%%%%%%%%%%%%%%%%%%%%%%%%%%%%
\newpage
\cleardoublepage
\ifLSRITRtutorial
		\phantom{u}
		\phantom{1}\vspace{6cm}
	\begin{center}
		\add[inline]{In your final hardback copy, replace this page with the signed exercise sheet.}
		\vspace{3cm}
		\add[inline]{!! If you delete this page, make sure that the page structure is maintained !!
		
		(when printed, (even) page numbers on \textit{left pages} must be on the left side, (odd) page numbers on \textit{right pages} on the right side}
		\vspace{3cm}
		\todo[inline,color=red!70]{Before modifying this document, READ THE INSTRUCTIONS AND GUIDELINES!
		
		(remove \textit{tutorial} flag in the first line of this file to remove template tutorial parts)}
	\end{center}
\else
	% in case you want to add the PDF directly add the task description in the include directory
	\includepdf[pages=1]{./include/task_desc.pdf}
\fi
\newpage

%%%%%%%%%%%%%%%%%%%%%%%%%%%%%%%%%%%%%%%%%%%%%%%%%%%%%%%%%%%%%%%
%%%%%%%%%%%%%%%%%%%%% Abstract %%%%%%%%%%%%%%%%%%%%%%%%%%%%%%%%
%%%%%%%%%%%%%%%%%%%%%%%%%%%%%%%%%%%%%%%%%%%%%%%%%%%%%%%%%%%%%%%
\topmargin5mm
\textheight220mm
\pagenumbering{arabic}
\phantom{u}
\begin{abstract}
As control engineering methods are applied to increasingly complex systems, data-driven approaches for system identification appear as a promising alternative to physics-based modeling.  Unfortunately, in many of these system the states cannot be measured which complicates the learning process and makes it difficult to quantify the epistemic and aleatory uncertainties of the unknown system. Particle Markov chain Monte Carlo (PMCMC) methods provide ways for the joint estimation of the uncertain dynamics and latent states. Combining PMCMC methods with scenario theory enables the computation of optimal input trajectories for unknown systems with latent states. However, deriving guarantees for these input trajectories is computationally complex.

This thesis proposes a novel method that combines PMCMC sampling with maximum mean discrepancy ambiguity sets as an alternative to scenario theory. The goal is to optimize the trajectory of a system while satisfying chance-constraints where the underlying distribtion of the data is unknown due to the unknown system and latent states. The PMCMC samples are used to create an estimate of the distribution and the ambiguity set is constructed around that estimate to account for inaccuracy due to the limited data. By using ambiguity sets, it is possible to optimize the problem over a set of distributions that we can obtain through our samples rather than optimizing over a single unknown distribution. This allows us to find a robust solution for the problem and, in constrast to the scenario theory, also allows for the tuning of the robustness level in the optimization process. This is advantagous since it can lead to a reduction of the costs in exchange for a higher risk of the solution violating the constraints. The effectiveness of this approach is demonstrated in numerical simulations.
%% German abstract (optional)
%\begin{center}	
%\normalsize \textbf{Zusammenfassung}\\
%\end{center}
%% Add the German abstract here
\end{abstract}
\newpage

%%%%%%%%%%%%%%%%%%%%%%%%%%%%%%%%%%%%%%%%%%%%%%%%%%%%%%%%%%%%%%%
%%%%%%%%%%%%%%%%% Preamble / Acknowledgments %%%%%%%%%%%%%%%%%%
%%%%%%%%%%%%%%%%%%%%%%%%%%%%%%%%%%%%%%%%%%%%%%%%%%%%%%%%%%%%%%%
\phantom{u}
\phantom{1}\vspace{6cm}
\begin{center}
%% Insert preamble or acknowledgments here or leave blank
\end{center}

\pagestyle{fancy}

%%%%%%%%%%%%%%%%%%%%%%%%%%%%%%%%%%%%%%%%%%%%%%%%%%%%%%%%%%%%%%%
%%%%%%%%%%%%%%%%%%% Table of Contents %%%%%%%%%%%%%%%%%%%%%%%%%
%%%%%%%%%%%%%%%%%%%%%%%%%%%%%%%%%%%%%%%%%%%%%%%%%%%%%%%%%%%%%%%
\tableofcontents 

%%%%%%%%%%%%%%%%%%%%%%%%%%%%%%%%%%%%%%%%%%%%%%%%%%%%%%%%%%%%%%%
%%%%%%%%%% Actual Content of the Report / Thesis %%%%%%%%%%%%%%
%%%%%%%%%%%%%%%%%%%%%%%%%%%%%%%%%%%%%%%%%%%%%%%%%%%%%%%%%%%%%%%
\ifLSRITRtutorial
	\input{./chapters/Tutorial.tex}
\fi
%_________Einleitung__________________________________
\chapter{Introduction}
\label{sec:introduction}

Reliable mathmatical models are a fundamental component of any model-based control application. However, finding such a model is often very difficult as there are many considerations that go into creating or estimating it, with even small deviations sometimes leading to big differences. Oftentimes, it is impossible to create such a model only based on the general knowledge of the system, such as the physics of the application. Because of this, data-driven modeling approaches that allow for derive such models based on previously collected data are gaining attention and their usefulness for optimal control applications is being explored.

In this chapter, the problem is introduced in Sec. \ref{Problem Statement} and various works that are connected to this topic are shown and summarized in \ref{Related Work}.

\section{Problem Statement} \label{Problem Statement}

Consider the general nonlinear discrete-time system of the form

\begin{subequations} \label{System equation}
\begin{equation}
\boldsymbol{x}_{t+1} = \boldsymbol{f} \left( \boldsymbol{x}_{t}, \boldsymbol{u}_t \right) + \boldsymbol{v}_{t}
\end{equation}
\begin{equation}
\boldsymbol{y}_{t} = \boldsymbol{g} \left( \boldsymbol{x}_{t}, \boldsymbol{u}_t \right) + \boldsymbol{w}_{t}
\end{equation}
\end{subequations}

with the state $\boldsymbol{x}_t \in \mathbb{R}^{n_x \in \mathbb{N}}$, the input $\boldsymbol{u}_t \in \mathbb{R}^{n_u \in \mathbb{N}}$, the output $\boldsymbol{y}_t \in \mathbb{R}^{n_y \in \mathbb{N}}$, the process noise $\boldsymbol{v}_{t} \in \mathbb{R}^{n_x}$, the measurement noise $\boldsymbol{w}_{t} \in \mathbb{R}^{n_y}$ and time $t \in \mathbb{Z}$. 

In our setting, only the output $\boldsymbol{y}$ is observable and the state transition function $\boldsymbol{f}(\cdot)$ and the observation function $\boldsymbol{g}(\cdot)$, as well as the distributions $\boldsymbol{\mathcal{V}}$ and $\boldsymbol{\mathcal{W}}$ of the process noise $\boldsymbol{v}$ and measurement noise $\boldsymbol{w}$ are unknown.

We assume that a dataset $\mathbb{D} = \left\{\boldsymbol{u}_{t}, \boldsymbol{y}_{t}\right\}_{t = \text{-}T:\text{-}1}$ containing the last $T \in \mathbb{N}$ measurements of the input $\boldsymbol{u}$ and output $\boldsymbol{y}$ is available.

We further assume that the structure of the model $\left\{\boldsymbol{f}_{\boldsymbol{\theta}}(\cdot), \boldsymbol{g}_{\boldsymbol{\theta}}(\cdot), \boldsymbol{\mathcal{V}}_{\boldsymbol{\theta}}, \boldsymbol{\mathcal{W}}_{\boldsymbol{\theta}}\right\}$ is known, for example through the underlying physics of the applications, and is dependent on a finite number of unknown parameters $\boldsymbol{\theta}$. In addition to that, the priors $p(\boldsymbol{\theta})$ and $p(\boldsymbol{x}_{\text{-}T})$ are available as well.

The objective is to minimize a given cost function 

\begin{equation} \label{cost function}
J_H = \sum_{t = 0}^H c(\boldsymbol{u}_t,  \boldsymbol{x}_t,  \boldsymbol{y}_t)
\end{equation}

over the horizon $H$ while satisfying the constraints 

\begin{equation} \label{constraints}
\boldsymbol{h}(\boldsymbol{u}_{0:H},  \boldsymbol{x}_{0:H},  \boldsymbol{y}_{0:H}) \leq \boldsymbol{0}
\end{equation}

with $\boldsymbol{h} \in \mathbb{R}^{n_c}$ being a vector of arbitrary deterministic function. As the states $\boldsymbol{x}_{0:H}$ are unknown to us and there are several uncertain factors in our system, the constraints are transformed into chance-constraints and since it is possible that $\boldsymbol{h}$ is impossible to satisfy for every possible $\boldsymbol{x}_{0:H}$, we also introduce a risk factor $\alpha \in [0, 1]$ that relaxes these constraints, turning them into

\begin{equation} \label{risk constraints}
P \left[ \text{max} (\boldsymbol{h}(\boldsymbol{u}_{0:H},  \boldsymbol{x}_{0:H},  \boldsymbol{y}_{0:H})) \leq 0 \right] \geq 1 - \alpha
\end{equation}

with $P$ being generally unknown.

\section{Related Work} \label{Related Work}

%From Robert's \emph{Learning-Based Optimal Control with Performance Guarantees for Unknown Systems with Latent States}\cite{Robert2024}:

The problem presented in Sec. \ref{Problem Statement} provides several challenges as the information we have available is very limited. While many methods to solve chance constraint problems exist, they often rely on samples from the distribution $P$ which is generally unknown in our problem. While we do have priors for the uncertain elements, samples drawn from this distributions do not allow as to quantify the system enough for practical applications. As such, the priors must be updated based on the input-output measurements $\mathbb{D}$ which generally results in an analytically intractable posterior distribution.

To draw samples from this distribution, methods such as particle Markov chain Monte Carlo (PMCMC) Methods \cite{Andrieu_10} can be used. This has recently been exploited for optimal control in \cite{Robert_24} , utilizing such a sampler to generate scenarios for the system and using these scenarios as a representation of the unknown distribution to formulate a deterministic optimal control problem by reformulating the chance-constraints. However, the usage of the scenarios in this paper comes with the drawback of the risk factor $\alpha$ not being part of the final optimal control problem (OCP) and the process of estimating it retroactively being quite ressource intensive.

As such, there is a need to find other methods that allow us to utilize the samples generated by PG sampler to reformulate the chance-constraints to find a distributionally robust solution without losing the risk factor in the process. 

As the difficulties with this can be traced back to the unknown distribution $P$, kernel distribution embeddings have been proposed in \cite{Adam_21} and \cite{Adam_22} to reformulate chance constrained control problems. However, these approaches work under the assumption that the states are known and does not allow for the latent states we are working with in our problem.

Another workaround that has been proposed is the use of ambiguity sets. Here, ambiguity sets are defined as a set of probability distributions that are within a certain radius under an appropriate distance function. This was used in \cite{Hota_19} with Wasserstein distance as the metric for the ambiguity set. It has however been proven rather difficult to efficiently construct a Wasserstein ambiguity set for problems. 

In contrast, the metric proposed in \cite{Yassine_22} allows for an efficient construction of an ambiguity set using a maximum mean discrepancy (MMD) metric combined with kernel approximation and can be applied to gneral non-linear and non-convex constraints.

The remainder of this paper is structured as follows. In chapter \ref{Technical Approach}, we review the methods used to solve the OCP. These methods are then tested and evaluated in chapter \ref{Evaluation} and further discussed in chapter \ref{Discussion}. Finally, the results are summarized and some concluding remarks are given in chapter \ref{Conclusion}.



%____________________________________________________
\chapter{Technical Background} \label{Technical Background}

In this chapter, an approach is outlined that allows us to effectively solve the chance-constraint problem defined in section \ref{Problem Statement}. In section \ref{PGibbs sampling} we explain how to draw samples from unknown system and use them to generate scenarios. These scenarios are then used in \ref{Scenario Approach} to reformulate and solve the OCP.

\section{Particle Markov Chain Monte Carlo Methods} \label{PGibbs sampling}

For practical applications, the known priors $p(\boldsymbol{\theta})$ and $p(\boldsymbol{x}_{\text{-}T})$ and the observations $\mathbb{D}$ must be used to infer the posterior $p(\boldsymbol{\theta}, \boldsymbol{x}_{\text{-}T:\text{-}1}\mid \mathbb{D})$. This is necessary since the repeated propagation of $p(\boldsymbol{x}_{\text{-}T})$ would otherwise cause an excessively large variance in $p(\boldsymbol{x}_{\text{-}1})$ making stochastic OCP infeasible. We utilize PMCMC methods to draw samples from this distribution. These methods were introduced in \cite{Andrieu_10} and will be summarized in this section.

We use Particle Gibbs (PG) to bypass the issue of an analytically intractable $p(\boldsymbol{\theta}, \boldsymbol{x}_{\text{-}T:\text{-}1}\mid \mathbb{D})$ by iteratively drawing samples from $p(\boldsymbol{\theta} \mid \boldsymbol{x}_{\text{-}T:\text{-}1}, \mathbb{D})$ and $p(\boldsymbol{x}_{\text{-}T:\text{-}1}\mid \boldsymbol{\theta}, \mathbb{D})$. We continually update the distributions with the previously drawn set, i.e. $\boldsymbol{x}_{\text{-}T:\text{-}1}^{[n]}$ is drawn from $p(\boldsymbol{x}_{\text{-}T:\text{-}1}^{[n]}\mid \boldsymbol{\theta}^{[n]}, \mathbb{D})$ and $\boldsymbol{\theta}^{[n+1]}$ is then drawn from $p(\boldsymbol{\theta}^{[n+1]}\mid \boldsymbol{x}_{\text{-}T:\text{-}1}^{[n]}, \mathbb{D})$. This is repeated until the desired number of samples has been achieved.

To ensure that the samples drawn through this method are an accurate representation of the distribution  $p(\boldsymbol{\theta}, \boldsymbol{x}_{\text{-}T:\text{-}1})$, additional steps are taken. For one, the first $N_p$ samples must be discarded as they are heavily reliant on the initialization and as such might show a strong bias. This burn in period should be chosen large enough so that this bias is no longer reflected in the samples. The samples should also be indepedendent of each other which is not given with this method as each $\boldsymbol{\theta}^{[n]}$ is dependent on $\boldsymbol{x}_{\text{-}T:\text{-}1}^{[n]}$ which in turn is dependent on $\boldsymbol{\theta}^{[n-1]}$. As such, measures must be taken to reduce the correlation between samples as much as possible. One approach to do this is thinning where only every $n_d$-th sample is used and the other samples are discarded. By increasing this parameter, the samples become more uncorrelated but there will also be a larger amount of samples created which leads to inefficiency.

\begin{algorithm}
	\caption{Scenario generation}\label{alg:PGibbs}
	\hspace*{\algorithmicindent} \textbf{Input}: Dataset $\mathbb{D}$, parametric model $\{\boldsymbol{f}_{\boldsymbol{\theta}}(\cdot), \boldsymbol{g}_{\boldsymbol{\theta}}(\cdot), \boldsymbol{\mathcal{V}}_{\boldsymbol{\theta}}, \boldsymbol{\mathcal{W}}_{\boldsymbol{\theta}}\}$, \\
	\hspace*{\algorithmicindent} \hspace*{\algorithmicindent} priors $p(\boldsymbol{\theta})$ and $p(\boldsymbol{x}_{\text{-}T})$, $N, H, T$ \\
	\hspace*{\algorithmicindent} \textbf{Output}: Scenarios $ \boldsymbol{\delta}^{[1:N]} = \{ \boldsymbol{\theta}, \boldsymbol{x}_0, \boldsymbol{v}_{0:H}, \boldsymbol{w}_{0:H}\}^{[1:N]}$
	\begin{algorithmic}[1]
		\For{$n = 1, \dots , N$}
			\State Sample $\{ \boldsymbol{\theta}, \boldsymbol{x}_{\text{-}T:\text{-}1} \}^{[n]}$ from $p\left( \boldsymbol{\theta}, \boldsymbol{x}_{\text{-}T:\text{-}1} \mid \mathbb{D} \right)$ using a PG sampler
			\For{$t = \text{-}1, \dots , H$}
				\State Sample $\boldsymbol{v}_t^{[n]}$ from $\boldsymbol{\mathcal{V}}_{\boldsymbol{\theta}^{[n]}}$
				\State Sample $\boldsymbol{w}_t^{[n]}$ from $\boldsymbol{\mathcal{W}}_{\boldsymbol{\theta}^{[n]}}$
			\EndFor
			\State $\boldsymbol{x}_0^{[n]} \gets \boldsymbol{f}_{\boldsymbol{\theta}^{[n]}} \left( \boldsymbol{x}_{\text{-} 1}^{[n]}, \boldsymbol{u}_{\text{-} 1} \right) + \boldsymbol{v}_{\text{-} 1}^{[n]}$
		\EndFor
	\end{algorithmic}
\end{algorithm}

The samples $\{\boldsymbol{\theta}, \boldsymbol{x}_{\text{-}T:\text{-}1}\}^{[1:N]}$ can be used to generate so-called scenarios $\boldsymbol{\delta}^{[1:N]}$, which are samples from the distribution $p(\boldsymbol{\theta}, \boldsymbol{x}_0, \boldsymbol{v}_{0:H}, \boldsymbol{w}_{0:H} \mid \mathbb{D})$ and represent possible future system behavior depending on $\boldsymbol{u}_{0:H}$. The generation of these scenarios is outlined in Algorithm \ref{alg:PGibbs}. The parameters $\boldsymbol{\theta}^{[n]}$ are obtained via PMCMC and through it we also know the system dynamics and noise distributions which can be used to draw samples of both the processing noise $\boldsymbol{v}_{0:H}$ and measurement noise $\boldsymbol{w}_{0:H}$, which can be seen in the lines 4 and 5 of the Algorithm. Those samples can then be combined with the $\boldsymbol{x}_{\text{-}T:\text{-}1}$, or more precisely $\boldsymbol{x}_{\text{-}1}$ to find the initial state $\boldsymbol{x}_{0}$ to complete the scenario $\boldsymbol{\delta} = \{ \boldsymbol{\theta}, \boldsymbol{x}_0, \boldsymbol{v}_{0:H}, \boldsymbol{w}_{0:H}\}$. How these scenarios can be used to find an optimal input $\boldsymbol{u}_{0:H}$ is described in the next section.


\section{Scenario Approach} \label{Scenario Approach}




\chapter{Chance-Constraint Optimization with Kernel Approximation} \label{Technical Approach}

In this chapter, an approach is outlined that allows us to effectively reformulate the chance-constraint problem defined in section \ref{Problem Statement} using the samples generated using the method described in chapter \ref{Technical Background}.

In section \ref{Scenario Approach} a method was presented that allows us to reformulate the chance-constraints as a determinstic OCP. However, the final problem no longer contains the risk factor $\alpha$ which is why in this section a alternative approach is presented that allows us to has been lo, a method that enables us to generate a finite number of scenarios $\boldsymbol{\delta}^{[1:N]}$ was presented. These scenarios can be used to formulate an OCP to find an optimal $\boldsymbol{u}$ or a control law $\boldsymbol{\pi}$. In this section, a method to use maximum mean discrepancy (MMD) ambiguity sets and kernel approximation to reformulate the OCP is proposed.

\section{MMD ambiguity sets} \label{SubSec:MMD}

As the underlying data distribution $P$ in the constraints \eqref{constraints} is unknown, we first expand the them to their distributionally robust counterpart in order to allow for the use of scenarios as an approximation of the distribution. For this, we consider $P$ as the worst case distribution within a set $\mathcal{P}$ of plausible distributions, the so-called ambiguity set. This gives us the new constraints

\begin{equation} \label{wc constraints}
\inf\limits_{\tilde{P} \in \mathcal{P}}\tilde{P} \left[ \text{max}(\boldsymbol{h}(\boldsymbol{u}_{0:H},  \boldsymbol{x}_{0:H},  \boldsymbol{y}_{0:H})) \leq 0 \right] \geq 1 - \alpha.
\end{equation}

\begin{figure}
\centering
\includegraphics[width= .4\textwidth]{AmbiguitySetDrawing}
\caption{Ambiguity Set $\mathcal{P}$}
\label{AmbiguityPic}
\end{figure}

To construct the set $\mathcal{P}$, a similarity measure is needed to provide a concrete comparison between various distributions $\tilde{P}$. Maximum mean discrepancy (MMD) \cite{Arthur_12} is able to do that by using the norm of the difference between the kernel mean embeddings (KME) $|| \mu_Q - \mu_{Q'} ||^2_{\mathcal{H}}$ of two distributions $Q$ and $Q'$ as a metric between two distributions. The KME are given as $\mu_Q = \int k(x, \cdot) \text{d}x$ with $k(x, \cdot) \in \mathcal{H}$ being the feature map of the kernel function $k$. The metric can then be rewritten as 

\begin{equation} \label{MMD Kernel}
\text{MMD}(Q, Q') = \text{E}_{z,z' \sim Q}[k(z,z')] + \text{E}_{z,z' \sim Q'}[k(z,z')] - 2\text{E}_{z\sim Q, z' \sim Q'}[k(z,z')]
\end{equation}

The MMD-based ambiguity set $\mathcal{P}$ is then constructed as the set of distributions $\tilde{P}$ in an $\varepsilon$ radius centered around the empirical distribution $P_N$ which is given through the scenarios $\boldsymbol{\delta}^{[1:N]}$. This gives us the set

\begin{equation} \label{ambiguity set}
\mathcal{P} =  \left\{ P : \text{MMD} (P, P_N) \leq \varepsilon \right\}.
\end{equation}

The radius $\varepsilon$ is chosen through constructing a bootstrap MMD ambiguity set as described in \cite{Yassine_22} and is outlined in Algorithm \ref{alg:Bootstrap}. This procedure requires a number of bootstrap samples $B$ to be chosen, as well as confidence level $\beta$. It then utilizes kernels $k(\boldsymbol{\delta}^{[i]}, \boldsymbol{\delta}^{[j]}) \in \mathbb{R}$ to define the (biased) MMD estimator as 

\begin{equation} \label{ambiguity set approx}
\widehat{\text{MMD}} (\tilde{P}, P_N) = \sum_{i,j = 1}^N k(\boldsymbol{\delta}^{[i]}, \boldsymbol{\delta}^{[j]}) + k(\tilde{\boldsymbol{\delta}}^{[i]}, \tilde{\boldsymbol{\delta}}^{[j]}) - 2 k(\boldsymbol{\delta}^{[i]}, \tilde{\boldsymbol{\delta}}^{[j]})
\end{equation}

with $\tilde{\boldsymbol{\delta}}^{[n]}, n = 1,...,N$, denoting a bootstrap sample of $P_N$ where samples are drawn with replacement from $\boldsymbol{\delta}^{[1:N]}$. Finally, $\widehat{\text{MMD}} (\tilde{P}, P_N)$ is calculated for all $B$ bootstrap samples, the results are saved in list and $\varepsilon$ is chosen as the $\textit{ceil}(B \beta)$-th element of the sorted list. 


\begin{algorithm}
	\caption{Bootstrap MMD ambiguity set}
	\label{alg:Bootstrap}
	\hspace*{\algorithmicindent} \textbf{Input}: Scenarios $ \boldsymbol{\delta}^{[1:N]} $, Number of bootstrap samples $B$, Confidence level $\beta$ \\
	\hspace*{\algorithmicindent} \textbf{Output}: Gram matrix $\boldsymbol{K}$, Radius of MMD ambiguity set $\varepsilon$
	\begin{algorithmic}[1]
		\State $\boldsymbol{K} \gets \textit{kernel}(\boldsymbol{\delta}, \boldsymbol{\delta})$
		\For{$m = 1, \dots , B$}
			\State $I \gets N$ numbers from $\{1, \dots N \}$ with replacement
			\State $K_x \gets \sum_{i,j = 1}^N K_{ij};$
			\State $K_y \gets \sum_{i,j \in I} K_{ij};$
			\State $K_{xy} \gets \sum_{j \in I} \sum_{i = 1}^N K_{ij}$;
			\State MMD$[m] \gets \frac{1}{N^2} \left( K_x + K_y - 2 K_{xy} \right) ;$
		\EndFor
		\State MMD $\gets$ \textit{sort}(MMD)
		\State $\varepsilon \gets$ MMD$\left[ \textit{ceil} (B \beta) \right]$
	\end{algorithmic}
\end{algorithm}

\section{Constraint Reformulation} \label{Constraint Reformulation}

With a given MMD ambiguity set $\mathcal{P}$, the feasible set of each constraint \ref{risk constraints} is given as

\begin{equation} \label{feasible set}
	Z :=  \left\{ \boldsymbol{u}_{0:H} \in \mathcal{U}^{H+1} : \inf\limits_{P \in \mathcal{P}}P \left[ \tilde{h}(\boldsymbol{u}_{0:H},  \boldsymbol{\delta}) \leq 0 \right] \geq 1 - \alpha \right\}.
\end{equation}

with $\tilde{h}(\boldsymbol{u}_{0:H},  \boldsymbol{\delta}) =  \text{max}(\boldsymbol{h}(\boldsymbol{u}_{0:H},  \boldsymbol{x}_{0:H},  \boldsymbol{y}_{0:H}))$.

We can now use the ambiguity set $\mathcal{P}$ defined in Sec. \ref{SubSec:MMD} with the radius $\varepsilon$ and the kernel matrix $\boldsymbol{K}$ to reformulate the feasible set. The matrix $\boldsymbol{K}$ contains the kernels of all possible combinations of samples. By following the steps described in \cite{Yassine_22}, we can obtain the new reformulated feasible set as

\begin{subequations}
  \begin{empheq}[right = \empheqrbrace, left= Z_i \coloneqq \empheqlbrace \boldsymbol{u}_{0:H} \in \mathcal{U}^{H+1} :]{align}
    & g_0 + \frac{1}{N}\sum_{n = 1}^N (\boldsymbol{K}\boldsymbol{\gamma})_n + \varepsilon \sqrt{\boldsymbol{\gamma}^\text{T}\boldsymbol{K}\boldsymbol{\gamma}} \leq t \alpha \\
    & [\tilde{h}(\boldsymbol{u}_{0:H},  \boldsymbol{\delta}^{[n]}) + t]_+ \leq g_0 + (\boldsymbol{K}\boldsymbol{\gamma})_n, \; n = 1,...,N \\
    & g_0 \in \mathbb{R}, \boldsymbol{\gamma} \in \mathbb{R}^N, t \in \mathbb{R}
  \end{empheq}
\end{subequations}

where $[\cdot]_+ = \text{max}(0, \cdot)$ denotes the max operator. The new set also contains new variables that have to be taken into account when solving the OCP. The variables $g_0$, $\gamma$ and $t$ are additional degrees of freedom can be incorperated into the OCP. The former are parameters of the RKHS function that was used to transform the problem into a kernel machine learning problem while $t$ has been introduced as a way to relax the risk factor $\alpha$.

We can then use this feasible set to formulate the OCP as

\begin{subequations}
\begin{align}
\begin{split}
\min\limits_{\boldsymbol{u}_{0:H},\boldsymbol{\gamma}, g_0, t' }  J_H(\boldsymbol{u}_{0:H})
\end{split}\\
\begin{split}
\text{s.t.}\; &\forall n = 1,...,N, \;  \forall t = 0,1,...,H
\end{split}\\
\begin{split}\label{systemc1}
&\boldsymbol{x}_{t+1}^{[n]} = \boldsymbol{f}_{\boldsymbol{\theta}^{[n]}} \left( \boldsymbol{x}_{t}^{[n]} , \boldsymbol{u}_t \right) + \boldsymbol{v}_{t}^{[n]}
\end{split}\\
\begin{split}\label{systemc2}
&\boldsymbol{y}_{t} = \boldsymbol{g}_{\boldsymbol{\theta}^{[n]}} \left( \boldsymbol{x}_{t}^{[n]}, \boldsymbol{u}_t \right) + \boldsymbol{w}_{t}^{[n]}
\end{split}\\
\begin{split}\label{systemc3}
 &\boldsymbol{u}_{0:H} \in Z(\boldsymbol{\gamma}, g_0, t')
\end{split}
\end{align}
\label{OCP_final}
\end{subequations}

As described in Sec. \ref{Problem Statement}, we are minimizing a cost function $J_H$. The system dynamics are included through the constraints \eqref{systemc1} and \eqref{systemc2} and must be fulfilled for all scenarios as well. Lastly, the input $u_{0:H}$ is restricted to the feasible set $Z$ in \eqref{systemc3} which is defined by the optimization variables $\boldsymbol{\gamma}$, $g_0$ and $t'$.

The optimization problem \eqref{OCP_final} is deterministic and can be solved with well known methods. 





\chapter{Evaluation} \label{Evaluation}

In this section, the proposed optimal control approach is implemented and its effectiveness tested and compared to the previous approach. The simulation setup is described in Sec. \ref{Setup}. The results of the OCP are shown in Sec. \ref{optimal control}. Afterwards, we analyse the results and performance in more detail in Sec. \ref{performance guarantees} by looking at the robustness of the solution and compare it to the commonly used scenario approach.

\section{Simulation Setup} \label{Setup}

We consider a system with the state transition function

\begin{equation}
\boldsymbol{f}(\boldsymbol{x}, u) = 
\begin{bmatrix}
0.8  x_1 - 0.5 x_2 \\
0.4 x_1 + 0.5 x_2 + u
\end{bmatrix}
\end{equation}

and the process noise distribution

\begin{equation}
\boldsymbol{v}_t \sim \mathcal{N} \left(\boldsymbol{0}, 
\begin{bmatrix}
0.03 & \text{-}0.004 \\
\text{-}0.004 & 0.01
\end{bmatrix}
\right).
\end{equation}

Both the state transition function and the process noise distribution are unknown to the user. Meanwhile, the observation function $g(\boldsymbol{x}, u) = x_1$ and measurement noise $w_t \sim \mathcal{N} (0, 0.1)$ are known. 


For the scenario generation, we consider a set containing $T = 2000$ input and output measurements of the true system for our dataset $\mathbb{D}$. These measurements are obtained with a random input trajectory $u \sim \mathcal{N} (0, 3)$ while starting from a random initial state $\boldsymbol{x}_{\text{-}T} \sim \mathcal{N} ([2, 2]^\text{T}, \boldsymbol{I}_2)$. To infer the state model parameters, the approach from \cite{Svensson_17} is used. It is assumed that $\boldsymbol{f}(\cdot)$ is a linear combination of $n_a$ basis functions $\boldsymbol{\varphi}(\boldsymbol{x}_t, u_t)$ and the process noise is normally distributed. As such, the state transition can be rewritten as

\begin{equation} \label{State transition}
\boldsymbol{x}_{t+1} = \boldsymbol{A} \boldsymbol{\varphi}(\boldsymbol{x}_t, u_t) + \boldsymbol{v}_{t}
\end{equation}

with the basis functions $\boldsymbol{\varphi} (\boldsymbol{x}, u) = \left[ x_1,  x_2,  u \right]^\text{T}$, the process noise $\boldsymbol{v}_{t} \sim \mathcal{N} (\boldsymbol{0}, \boldsymbol{Q})$ and the unknown parameters $\boldsymbol{\theta}$ consisting of $\boldsymbol{A}$ and $\boldsymbol{Q}$. An inverse Wishart  prior with $l$ degrees of freedom and positive definite scale matrix $\Lambda$ is assumed for the matrix $\boldsymbol{Q}$. For the matrix $\boldsymbol{A}$ matrix normal prior with mean matrix $\boldsymbol{M} = \boldsymbol{0}$, right covariance $\boldsymbol{U} = \boldsymbol{Q}$ and left covariance matrix $\boldsymbol{V} \in \mathbb{R}^{n_a \times n_a}.$ For the estimation of the posterior disdribution with the PG sampler, we scale the basisvector with the weights $\left[ 0.1,  0.1,  1 \right]^\text{T}$ and for the prior the weights are chosen as $\boldsymbol{V} = 10 \boldsymbol{I}_5$.

We use gaussian kernels $k(x,y) = \text{exp}\left(\text{-}\frac{1}{2\sigma^2} ||x - y||_2^2 \right)$ with the bandwidth $\sigma$ set individually for all random parameters $\left\{\boldsymbol{x}_0^{[k]}, \boldsymbol{v}_{0:H}^{[k]}, w_{0:H}^{[k]},  \boldsymbol{A}^{[k]}\right\}$ via the median heuristic \cite{Damien_18} and scaled with the factors $\left[ 1.5, 5, 5, 1 \right]^\text{T}$. As such, the elements of the Gram matrix $\boldsymbol{K} \in \mathbb{R}^{N \times N }$ are defined as

\begin{equation} \label{Kernel equation}
K_{ij} = k_{A}(\boldsymbol{A}^{[i]}, \boldsymbol{A}^{[j]})  k_{\mathcal{X}}(\boldsymbol{x}_0^{[i]}, \boldsymbol{x}_0^{[j]})    k_{\mathcal{V}^H}(\boldsymbol{v}_{0:H}^{[i]}, \boldsymbol{v}_{0:H}^{[j]})  k_{\mathcal{W}^H}(\boldsymbol{w}_{0:H}^{[i]}, \boldsymbol{w}_{0:H}^{[j]}).
\end{equation}


\section{Optimal Control with Constrained Output} \label{optimal control}

In the following, we show how well the proposed optimal control approach works when applied to a OCP with constrained output by putting it side by side with the solution of the same problem where we have used the Scenario approach which implements the chance-constraints by ensuring that the constraints are satisfied for every scenario $n \in \mathbb{N}_{\leq N}$.

For this simulation, we are scenarios that have been generated using the PG sampler. To this end, $9170$ samples were created and the first $N_p = 1000$ were discarded as training samples and the remaining samples were once again thinned with $n_d = 30$. The remaining $N = 200$ samples are then used as scenarios for the OCPs.

For the cost function, we consider a simple quadratic cost $J_H = \sum_{t = 0}^H u_t^2$ over the horizon $H = 40$. For constraints, we consider the input-constraint $\left| u \right| \leq 10$ as well as the temporarily active output-constraints $y_{10:20} \leq \text{-} 10$ and $10 \leq y_{30:40}$. The risk level $\alpha$ is fixed at $0.1$ for this experiment and $\epsilon$ is chosen through Algorithm \ref{alg:Bootstrap}.

The OCP can then be formulated as described in \ref{OCP final}. Since the problem has been chosen as convex in this example, a solution for both the scenario and kernel approach can be found easily by using a convex solver. The results of an exemplary run is shown in Fig. \ref{ScenarioKernelComparison}. 
\iffalse
\begin{figure}[htb]
\centering
\subfigure[Scenario Approach]{
   \includegraphics[width=0.4\textwidth] {pics/Scenario_plot.png}
   \label{fig:subfigScenario}
 }
\quad % puts next subfigure right next to the previous subfigure
\subfigure[Kernel Approach]{
   \includegraphics[width=0.4\textwidth] {pics/Kernel_plot.png}
   \label{fig:subfigKernel}
 }
\caption{Example of the optimal control with known basis functions for scenario approach (left) and kernel approach (right). The red and blue area show the output constraints. The gray area encompasses the 200 scenarios that were used in the optimization with the orange line being the average. The blue line is one realization of the true output.}
\label{ScenarioKernelComparison}
\end{figure}
\fi

\begin{figure}[htb]
\centering
\subfigure[Scenario Approach]{
\pgfplotsset{width=.47\textwidth, compat = 1.18, 
			height = .4\textwidth, grid= major, 
			legend cell align = left, ticklabel style = {font=\scriptsize},
			every axis label/.append style={font=\scriptsize},
			legend style = {font=\scriptsize},
        }
		\def\file{data/Scenario_K200_S2.txt}
		
		\centering
		\begin{tikzpicture}
		\begin{axis}[
		grid=none,
		xmin=0, xmax=40,
		ymin=-14, ymax=14,
		xtick={0, 10, 20, 30, 40},
		ytick={-10, -5, 0, 5, 10},
		ylabel=$y$, xlabel=$t$,
		set layers=standard,
		reverse legend,
		legend style={font=\scriptsize, at={(1,0)},anchor=south east, row sep=2pt},
		ylabel shift = -6 pt]
		
		\addplot[name path=A, forget plot, thick, opacity=0.2] table[x=t,y=y_opt_max]{\file};
		\addplot[name path=B, thick, opacity=0.2] table[x=t,y=y_opt_min]{\file};
		\tikzfillbetween[of=A and B]{opacity=0.2};
		\addlegendentry{$\left\{y_{0:H}^{[1:N]} \right\}$}
		
		\addplot[ultra thick,black!20!green] table[x=t,y=y_pred]{\file};
		\addlegendentry{$\frac{1}{N}\sum\limits_{n=1}^N y_{0:H}^{[n]}$}
		
		\addplot[ultra thick, blue] table[x=t,y=y_true]{\file};
		\addlegendentry{$y_{0:H}$}
		
		\draw [fill=red, fill opacity=0.2,red, opacity=0.2] (10,15) rectangle (20,-10); 
		\draw [fill=red, fill opacity=0.2,red, opacity=0.2] (30,10) rectangle (40,-15); 
		\end{axis}
		\end{tikzpicture}
 }
%\quad % puts next subfigure right next to the previous subfigure
\subfigure[Kernel Approach]{
\pgfplotsset{width=.47\textwidth, compat = 1.18, 
			height = .4\textwidth, grid= major, 
			legend cell align = left, ticklabel style = {font=\scriptsize},
			every axis label/.append style={font=\scriptsize},
			legend style = {font=\scriptsize},
        }
		\def\file{data/Kernel_K200_Alpha01_S2.txt}
		
		\centering
		\begin{tikzpicture}
		\begin{axis}[
		grid=none,
		xmin=0, xmax=40,
		ymin=-14, ymax=14,
		xtick={0, 10, 20, 30, 40},
		ytick={-10, -5, 0, 5, 10},
		%ylabel=$y$, 
		xlabel=$t$,
		set layers=standard,
		reverse legend,
		legend style={font=\scriptsize, at={(1,0)},anchor=south east, row sep=2pt},
		ylabel shift = -6 pt]
		
		\addplot[name path=A, forget plot, thick, opacity=0.2] table[x=t,y=y_opt_max]{\file};
		\addplot[name path=B, thick, opacity=0.2] table[x=t,y=y_opt_min]{\file};
		\tikzfillbetween[of=A and B]{opacity=0.2};
		\addlegendentry{$\left\{y_{0:H}^{[1:N]} \right\}$}
		
		\addplot[ultra thick,black!20!green] table[x=t,y=y_pred]{\file};
		\addlegendentry{$\frac{1}{N}\sum\limits_{n=1}^N y_{0:H}^{[n]}$}
		
		\addplot[ultra thick, blue] table[x=t,y=y_true]{\file};
		\addlegendentry{$y_{0:H}$}
		
		\draw [fill=red, fill opacity=0.2,red, opacity=0.2] (10,15) rectangle (20,-10); 
		\draw [fill=red, fill opacity=0.2,red, opacity=0.2] (30,10) rectangle (40,-15); 
		\end{axis}
		\end{tikzpicture}
 }
\caption{Example of the optimal control with known basis functions for scenario approach (left) and kernel approach (right). The red areas show the output constraints. The gray area encompasses the 200 scenarios that were used in the optimization with the green line being the average. The blue line is one realization of the true output.}
\label{ScenarioKernelComparison}
\end{figure}



The figure includes the two plots for the scenario and kernel approach and shows the output $y$ of their respective OCPs. The graphs shows the spread of the $N = 200$ trajectories that are generated when the input $\boldsymbol{u}_{0:H}$ is applied to the scenarios that were used to find the optimal input. On top that, it also shows of the mean of these trajectories and true output. Where the two graphs differ however is to what extend the solutions fulfill the constraints. By its definition, the scenario approach requires all scenarios to fulfill the constraints which can be seen in the solution. While the gray area touches the min and max constraints in several places, it never violates the constraints. The kernel approach on the other hand has a risk factor $\alpha$ built in which allows for a number of scenarios to violate the constraints as long a sufficient number satisfies them. This can be seen in the constraint still being met by the majority of the scenarios and only the trajectory of a small number of scenarios actively overlapping with the marked area. As a result, a solution with a lower cost was found in exchange for an increased risk of the true output violating one or more of the constraints.




\section{Robustness} \label{performance guarantees}

The biggest advantage that this kernel approximation proves compared to the scenario approach is the adjustable risk factor. As described in Sec. \ref{Sec:CCOKernel}, this method includes a parameter $\alpha \in [0, 1]$ which can be chosen depending on how successful the final solution is supposed to be when it comes to satisfying the constraints in future scenarios. 

In this section, this parameter is tested by running the same problem setup as was used in Sec. \ref{optimal control} for different values of $\alpha$ as well as an increasing number of samples $N$ and testing how well the solution holds up for future scenarios.

Similar to Sec. \ref{optimal control}, a number of scenarios are generated with Algorithm \ref{alg:PGibbs}. From this set scenarios, a small subset is then taken and used to formulate several OCPs as was already described in Sec. \ref{optimal control}. The OCPs are then solved and the resulting optimal input $\boldsymbol{u}_{0:H}$ is then used on $N = 2000$ more independant scenarios from the same system to test how well this solution holds up. For each of the 2000 scenarios, the output is calculated and compared to the constraints that were used in the OCP to check whether or not they are fulfilled. This process is then repeated for various numbers of scenarios. It is done for 1, 5, 10 and 25 samples and then increased with a step size of 25 until a $N = 250$ samples are used in the optimization.

\iffalse
\begin{figure}[htb]
\centering
\includegraphics[width=0.7\textwidth]{pics/robustness_plot.png}
\caption{Percentage of scenarios where $u_{0:H}$ is a viable solution. The blue line shows the result of the scenario approach while the other lines are for the Kernel approach with various values of $\alpha$}
\label{fig:robustness_plot}
\end{figure}
\fi

\begin{figure}[t]
		\pgfplotsset{width=13cm, compat = 1.18, 
			height = 9cm, grid= major, 
			legend cell align = left, ticklabel style = {font=\scriptsize},
			every axis label/.append style={font=\scriptsize},
			legend style = {font=\scriptsize},
        }
		\def\file{data/AlphaTest_K250_MultipleConstraints_S2_Alpha02.txt}
		
		\centering
		\begin{tikzpicture}
		\begin{axis}[
		grid=both,
		xmin=1, xmax=250,
		ymin=0, ymax=100,
		xtick={0, 50, 100, 150, 200, 250},
		ytick={0, 20, 40, 60, 80, 100},
		ylabel={Constraints Satisfied [\%]}, xlabel=$N$,
		set layers=standard,
		reverse legend,
		legend style={font=\scriptsize, at={(1,0)},anchor=south east, row sep=2pt},
		ylabel shift = -6 pt]
		
		\addplot[ultra thick,black!20!red] table[x=K,y=Kernel02]{\file};
		\addlegendentry{Kernel Approach ($\alpha = 0.2$)}
		
		\addplot[ultra thick,black!20!orange] table[x=K,y=Kernel015]{\file};
		\addlegendentry{Kernel Approach ($\alpha = 0.15$)}

		\addplot[ultra thick,black!20!yellow] table[x=K,y=Kernel01]{\file};
		\addlegendentry{Kernel Approach ($\alpha = 0.1$)}

		\addplot[ultra thick,black!20!green] table[x=K,y=Kernel005]{\file};
		\addlegendentry{Kernel Approach ($\alpha = 0.05$)}

		\addplot[ultra thick,black!20!blue] table[x=K,y=Scenario]{\file};
		\addlegendentry{Scenario Approach}
		\end{axis}
		\end{tikzpicture}
		\vspace*{-0.4cm}
		
		\caption{Percentage of scenarios where $u_{0:H}$ is a viable solution. The blue line shows the result of the scenario approach while the other lines are for the kernel approach with various values of $\alpha$.}
		\label{fig:robustness_plot}
\end{figure}


In Fig. \ref{fig:robustness_plot} the results of this simulation are shown. The percentage of scenarios that fulfill the constraints is plotted over the number of samples used in the initial optimization which range from $N = 1$ to 250. The various $\alpha$ values are shown as separate lines. Initially, all five plots show very similar results. This can be explained by the fact that at such a low number of scenarios cannot accurately represent the distribution. As the number of scenarios is increased, the approximation of the distribution becomes better leading to a higher percentage of scenarios where the constraints are satisfied.

After around 10 scenarios, the plots start diverging for the first time. While the scenario approach and the plots with smaller $\alpha$ values are very similar, the lines that represent larger $\alpha$ values are starting to display worse percentage of cases that satisfy the constraints. This trend continues as the number of scenarios used in the optimization keeps increasing. While the scenario approach keeps increasing, the kernel approach seems to converge to a significantly lower level of robustness based on the selection of the parameter $\alpha$. This shows that through $\alpha$ we are able to control how distributionally robust the solution is.

 
\iffalse
\section{Computation Time} \label{computation time}

While the previous sections focused on comparing the results of the Scenario approach to the kernel approach, the runtime of both methods is another important factor that needs to be considered when choosing which algorithm is best suited for a specific problem. As such, Fig. \ref{fig:runtime_plot} shows the runtime of both scenario approach and kernel approach over the number of scenarios that are used to formulate the OCPs. The parameters and constraints are chosen the same as in Sec. \ref{optimal control}. When looking at the figure, it quickly becomes apparent that while both curves start around the same level, the kernel approach runtime increases significantly faster and with more of a curvature than the scenario approach. By the end of the plot, the kernel approach already takes about 5 times as long as the scenario approach. This shows that at least when it comes to finding a solution quickly, the Scenario approach is still the better choice.

\begin{figure}[htb]
\centering
\includegraphics[width=0.6\textwidth]{pics/computationtime_plot.png}
\caption{Runtime over number of scenarios $N$ for scenario and kernel approach}
\label{fig:runtime_plot}
\end{figure}
\fi
%\chapter{Discussion} \label{Discussion}

%Discuss and explain your results. Show how they support your thesis (or, if they do not, give a convincing explanation). It is important to separate objective facts clearly from their discussion (which is bound to contain subjective opinion). If the reader does not understand your results, reconsider if you have managed to extract the core information and explain it in a straightforward way.

%_______________________________________________

%_____Zusammenfassung, Ausblick_________________________________
\chapter{Conclusion}

%Do not just leave it at the discussion: discuss what you/the reader can learn from the results. Draw some real conclusions. Separate discussion/interpretation of the results clearly from the conclusions you draw from them. (So-called ``conclusion creep'' tends to upset reviewers. It means surrendering your scientific objectivity.) Identify all shortcomings/limitations of your work, and discuss how they could be fixed (``future work''). It is not a sign of weakness of your work if you clearly analyze and state the limitations. Informed readers will notice them anyway and draw their own conclusions if not addressed properly.

%\vspace{\baselineskip}
%Recap: do not stick to this structure at all cost. Also, remember that the thesis must be:

%\begin{itemize}
	%\item honest, stating clearly all limitations
	%\item self--contained, do not write just for the locals, do not assume that the reader has read the same literature as you, do not let the reader work out the details for themselves
%\end{itemize}

%This chapter is followed by the list of figures and the bibliography. If you are using acronyms, listing them (with the expanded full name) before the bibliography is also a good idea. The acronyms package helps with consistency and an automatic listing.



%%%%%%%%%%%%%%%%%%%%%%%%%%%%%%%%%%%%%%%%%%%%%%%%%%%%%%%%%%%%%%%
%%%%%%%%%%%%%%%%%%%%%%%% Appendix %%%%%%%%%%%%%%%%%%%%%%%%%%%%%
%%%%%%%%%%%%%%%%%%%%%%%%%%%%%%%%%%%%%%%%%%%%%%%%%%%%%%%%%%%%%%%
\appendix

% Add optional appendix here
\input{./chapters/Appendix.tex}

% List of figures
\iffalse
\cleardoublepage
\phantomsection
\addcontentsline{toc}{chapter}{\listfigurename} 
\listoffigures 	

% List of tables
\cleardoublepage
\phantomsection
\addcontentsline{toc}{chapter}{\listtablename} 
\listoftables
\fi

% Acronyms and notation
% -> see include/gloss.tex
\ifdefined\AddMyGloss
    \AddMyGloss 
\fi

% Bibliography
\cleardoublepage
\phantomsection
\addcontentsline{toc}{chapter}{Bibliography}
\bibliography{./refs/mybib}
\bibliographystyle{alphaurl}

% License
\cleardoublepage
\chapter*{License}
\markright{LICENSE}
This work is licensed under the Creative Commons Attribution 3.0 Germany License. To view a copy of this license, visit \href{http://creativecommons.org/licenses/by/3.0/de/}{http://creativecommons.org} or send a letter to Creative Commons, 171 Second Street, Suite 300, San Francisco, California 94105, USA.

% Todo list
% This MUST be empty and/or removed in the final version!
\listoftodos
\end{document}
