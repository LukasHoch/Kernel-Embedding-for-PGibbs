%_____Zusammenfassung, Ausblick_________________________________
\chapter{Conclusion} \label{Conclusion}

This work presented an alternative approach to scenario theory for the use of samples generated using PMCMC methods. We've combined the samples drawn from the posterior distribution for the trajectories of an unknown system using a PG sampler with ambiguity sets to find a robust solution to our problem. We have done this by defining MMD ambiguity sets with the use of kernel embeddings. These sets were then used to reformulate a chance-constrained OCP while including a parameter for the permitted failure probability. This problem can then be solved using well-known methods.

The viability of this approach has been shown in several simulations where the methods were tested for an unknown linear system with a constrained output. The predicted output has been plotted and compared with the real output for the OCPs that were generated using scenario theory and MMD ambiguity sets. The effect of the permitted failure probability has also been shown for various numbers of scenarios being used when formulating the OCP. These results have demonstrated that the failure percentage is not converging to the selected parameter. The potential benefits of this approach have been shown in a simulation where a risky path leads to a more cost-effective solution, and the choice of the risk factor has been shown to have a major effect on what path the solver selects. Finally, the approach has been tested on a nonlinear system.


%Do not just leave it at the discussion: discuss what you/the reader can learn from the results. Draw some real conclusions. Separate discussion/interpretation of the results clearly from the conclusions you draw from them. (So-called ``conclusion creep'' tends to upset reviewers. It means surrendering your scientific objectivity.) Identify all shortcomings/limitations of your work, and discuss how they could be fixed (``future work''). It is not a sign of weakness of your work if you clearly analyze and state the limitations. Informed readers will notice them anyway and draw their own conclusions if not addressed properly.

%\vspace{\baselineskip}
%Recap: do not stick to this structure at all cost. Also, remember that the thesis must be:

%\begin{itemize}
	%\item honest, stating clearly all limitations
	%\item self--contained, do not write just for the locals, do not assume that the reader has read the same literature as you, do not let the reader work out the details for themselves
%\end{itemize}

%This chapter is followed by the list of figures and the bibliography. If you are using acronyms, listing them (with the expanded full name) before the bibliography is also a good idea. The acronyms package helps with consistency and an automatic listing.
