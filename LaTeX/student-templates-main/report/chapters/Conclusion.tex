%_____Zusammenfassung, Ausblick_________________________________
\chapter{Conclusion}

%Do not just leave it at the discussion: discuss what you/the reader can learn from the results. Draw some real conclusions. Separate discussion/interpretation of the results clearly from the conclusions you draw from them. (So-called ``conclusion creep'' tends to upset reviewers. It means surrendering your scientific objectivity.) Identify all shortcomings/limitations of your work, and discuss how they could be fixed (``future work''). It is not a sign of weakness of your work if you clearly analyze and state the limitations. Informed readers will notice them anyway and draw their own conclusions if not addressed properly.

%\vspace{\baselineskip}
%Recap: do not stick to this structure at all cost. Also, remember that the thesis must be:

%\begin{itemize}
	%\item honest, stating clearly all limitations
	%\item self--contained, do not write just for the locals, do not assume that the reader has read the same literature as you, do not let the reader work out the details for themselves
%\end{itemize}

%This chapter is followed by the list of figures and the bibliography. If you are using acronyms, listing them (with the expanded full name) before the bibliography is also a good idea. The acronyms package helps with consistency and an automatic listing.
