\chapter{Technical Approach}

\section{Problem Formulation} \label{Problem Formulation}

Consider the general nonlinear discrete-time system of the form

\begin{subequations} \label{System equation}
\begin{equation}
\boldsymbol{x}_{t+1} = \boldsymbol{f} \left( \boldsymbol{x}_{t}, \boldsymbol{u}_t \right) + \boldsymbol{v}_{t}
\end{equation}
\begin{equation}
\boldsymbol{y}_{t} = \boldsymbol{g} \left( \boldsymbol{x}_{t}, \boldsymbol{u}_t \right) + \boldsymbol{w}_{t}
\end{equation}
\end{subequations}

with the state $\boldsymbol{x} \in \mathbb{R}^{n_x \in \mathbb{N}}$, the input $\boldsymbol{u} \in \mathbb{R}^{n_u \in \mathbb{N}}$, the output $\boldsymbol{y} \in \mathbb{R}^{n_y \in \mathbb{N}}$ and time $t \in \mathbb{Z}$. 

In our setting, the state $\boldsymbol{x}$ is not fully observable and the state transition function $\boldsymbol{f}(\cdot)$ and the observation function $\boldsymbol{g}(\cdot)$, as well as the distributions $\boldsymbol{\mathcal{V}}$ and $\boldsymbol{\mathcal{W}}$ of the process noise $\boldsymbol{v}$ and measurement noise $\boldsymbol{w}$ are unknown.

We assume that a dataset $\mathbb{D} = \left\{\boldsymbol{u}_{t}, \boldsymbol{y}_{t}\right\}_{t = \text{-}T:\text{-}1}$ containing the last $T \in \mathbb{N}$ measurements of the input $\boldsymbol{u}$ and output $\boldsymbol{y}$.

We further assume, that the structure of the model depending on a finite number of parameters $\boldsymbol{\theta}$ is known as $\left\{\boldsymbol{f}_{\boldsymbol{\theta}}(\cdot), \boldsymbol{g}_{\boldsymbol{\theta}}(\cdot), \boldsymbol{\mathcal{V}}_{\boldsymbol{\theta}}, \boldsymbol{\mathcal{W}}_{\boldsymbol{\theta}}\right\}$. In addition to that, the priors $p(\boldsymbol{\theta})$ and $p(\boldsymbol{x}_{\text{-}T})$ are available as well.

The objective is to minimize a given cost function 

\begin{equation} \label{cost function}
J_H = \sum_{t = 0}^H c(\boldsymbol{u}_t,  \boldsymbol{x}_t,  \boldsymbol{y}_t)
\end{equation}

while satisfying constraints 

\begin{equation} \label{constraints}
\boldsymbol{h}(\boldsymbol{u}_{0:H},  \boldsymbol{x}_{0:H},  \boldsymbol{y}_{0:H}) \leq \boldsymbol{0}
\end{equation}

with $\boldsymbol{h} \in \mathbb{R}^{n_c}$ being a vector of arbitrary deterministic function. As it might be impossible to guarantee that $\boldsymbol{h}$ is satisfied for every possible scenario, we also introduce a risk factor $\alpha$ that relaxes these constraints, turning them into

\begin{equation} \label{risk constraints}
P_0 \left[ h_i(\boldsymbol{u}_{0:H},  \boldsymbol{x}_{0:H},  \boldsymbol{y}_{0:H}) \leq 0 \right] \geq 1 - \alpha, \; \forall i = 1,...,n_c
\end{equation}

with $h_i$ being the $i$-th element of $\boldsymbol{h}$ and $P_0$ being generally unknown.



\section{Particle Markov Chain Monte Carlo Methods} \label{PGibbs sampling}

For practical applications, the known priors $p(\boldsymbol{\theta})$ and $p(\boldsymbol{x}_{\text{-}T})$ must be used to infer the posterior $p(\boldsymbol{\theta}, \boldsymbol{x}_{\text{-}T:\text{-}1}| \mathbb{D})$ using the observations $\mathbb{D}$. This is necassary since the repeated propagation of $p(\boldsymbol{x}_{\text{-}T})$ would otherwise cause an excessively large variance in $p(\boldsymbol{x}_{\text{-}1})$ making stochastic OCP infeasible. One way to draw samples from this distribution are particle Markov chain Monte Carlo (PMCMC) methods which is explained in \cite{Andrieu_10} and will be summarized in this section.

While in general, the inference of the psoterior PMCMC methods $p(\boldsymbol{\theta}, \boldsymbol{x}_{\text{-}T:\text{-}1}| \mathbb{D})$ is analytically intractable, PMCMC bypass this issue by iteratively drawing samples from $p(\boldsymbol{\theta}| \boldsymbol{x}_{\text{-}T:\text{-}1}, \mathbb{D})$ and $p(\boldsymbol{x}_{\text{-}T:\text{-}1}| \boldsymbol{\theta}, \mathbb{D})$ while updating the distributions with the previously drawn set, i.e. $\boldsymbol{x}_{\text{-}T:\text{-}1}^{[n]}$ is drawn from $p(\boldsymbol{x}_{\text{-}T:\text{-}1}^{[n]}| \boldsymbol{\theta}^{[n]}, \mathbb{D})$ and $\boldsymbol{\theta}^{[n+1]}$ is then drawn from $p(\boldsymbol{\theta}^{[n+1]}| \boldsymbol{x}_{\text{-}T:\text{-}1}^{[n]}, \mathbb{D})$. This is repeated until the desired number of samples has been achieved.

To ensure that the samples drawn through this method are an accurate representation of the distribution  $p(\boldsymbol{\theta}, \boldsymbol{x}_{\text{-}T:\text{-}1})$, additional steps are taken. For one, the first $N_p$ samples must be discarded as they are heavily reliant on the initialization and as such might show a strong bias that isn't rpesent in the real distribution. In regards to the length of this burn-in period, it depends on the system. The samples should also be indepedendant of each other which is not given with this method as each $\boldsymbol{\theta}^{[n]}$ is dependant on $\boldsymbol{x}_{\text{-}T:\text{-}1}^{[n]}$. As such, measures must be taken to reduce the correlation between samples as much as possible. One approach to do this is thinning where only every $n_d$-th sample is used and the other samples are discarded. By increasing this parameter, the samples become more uncorrelated but there will also be a larger amount of samples created which leads to inefficiency.

\begin{algorithm}
	\caption{Scenario generation}\label{alg:PGibbs}
	\hspace*{\algorithmicindent} \textbf{Input}: Dataset $\mathbb{D}$, parametric model $\{\boldsymbol{f}_{\boldsymbol{\theta}}(\cdot), \boldsymbol{g}_{\boldsymbol{\theta}}(\cdot), \boldsymbol{\mathcal{V}}_{\boldsymbol{\theta}}, \boldsymbol{\mathcal{W}}_{\boldsymbol{\theta}}\}$, \\
	\hspace*{\algorithmicindent} \hspace*{\algorithmicindent} priors $p(\boldsymbol{\theta})$ and $p(\boldsymbol{x}_{\text{-}T})$, $N, H, T$ \\
	\hspace*{\algorithmicindent} \textbf{Output}: Scenarios $ \boldsymbol{\delta}^{[1:N]} = \{ \boldsymbol{\theta}, \boldsymbol{x}_0, \boldsymbol{v}_{0:H}, \boldsymbol{w}_{0:H}\}^{[1:N]}$
	\begin{algorithmic}[1]
		\For{$n = 1, \dots , N$}
			\State Sample $\{ \boldsymbol{\theta}, \boldsymbol{x}_{\text{-}T:\text{-}1} \}^{[n]}$ from $p\left( \boldsymbol{\theta}, \boldsymbol{x}_{\text{-}T:\text{-}1} \mid \mathbb{D} \right)$ using a PMCMC method
			\For{$t = \text{-}1, \dots , H$}
				\State Sample $\boldsymbol{v}_t^{[n]}$ from $\boldsymbol{\mathcal{V}}_{\boldsymbol{\theta}^{[n]}}$
				\State Sample $\boldsymbol{w}_t^{[n]}$ from $\boldsymbol{\mathcal{W}}_{\boldsymbol{\theta}^{[n]}}$
			\EndFor
			\State $\boldsymbol{x}_0^{[n]} \gets \boldsymbol{f}_{\boldsymbol{\theta}^{[n]}} \left( \boldsymbol{x}_{\text{-} 1}^{[n]}, \boldsymbol{u}_{\text{-} 1} \right) + \boldsymbol{v}_{\text{-} 1}^{[n]}$
		\EndFor
	\end{algorithmic}
\end{algorithm}

The samples $\{\boldsymbol{\theta}, \boldsymbol{x}_{\text{-}T:\text{-}1}\}^{[1:N]}$ can then be used to draw samples from the distribution $p(\boldsymbol{\theta}, \boldsymbol{x}_0, \boldsymbol{v}_{0:H}, \boldsymbol{w}_{0:H} | \mathbb{D})$ as can be seen in Algorithm \ref{alg:PGibbs}. $\boldsymbol{\theta}^{[n]}$ is already given and through it we also know the system dynamics and noise distributions which can be used to draw samples of both the processing noise $\boldsymbol{v}_{0:H}$ and measurement noise $\boldsymbol{w}_{0:H}$. Those can then be combined with the $\boldsymbol{x}_{\text{-}T:\text{-}1}$, or more precisely $\boldsymbol{x}_{\text{-}1}$ to find the initial state $\boldsymbol{x}_{0}$ to complete the scenario $\boldsymbol{\delta}$. How these scenarios can be used to find an optimal input $\boldsymbol{u}_{0:H}$ is described in the next section.


\section{Chance-Constraint Optimization with Kernel Approximation} \label{Sec:CCOKernel}

In the previous section, a method that enables us to generate a finite number of scenarios $\boldsymbol{\delta}^{[1:N]}$ was presented. These scenarios can be used to formulate an OCP to find an optimal $\boldsymbol{u}$ or a control law $\boldsymbol{\pi}$. In this section, a method to use maximum mean discrepancy (MMD) ambiguity sets and Kernel approximation to reformulate the OCP is proposed.


\subsection{MMD ambiguity sets} \label{SubSec:MMD}

As the underlying data distribution $P_0$ in the constraints \ref{constraints} is unknown, we first expand the it to its distributionally robust counterpart in order to allow for the use of scenarios as an approximation of the distribution. For this, we consider $P_0$ as the worst case distribution within a set $\mathcal{P}$ of plausible distributions, the so-called ambiguity set. This gives us the new constraints

\begin{equation} \label{wc constraints}
\inf\limits_{P \in \mathcal{P}}P \left[ h(\boldsymbol{u}_{0:H},  \boldsymbol{x}_{0:H},  \boldsymbol{y}_{0:H}) \leq 0 \right] \geq 1 - \alpha.
\end{equation}

The ambiguity set is constructed as the set of distributions $P$ in an $\varepsilon$ radius centered around the empirical distribution $\hat{P}_N$ which is given through the scenarios $\boldsymbol{\delta}^{[1:N]}$. The distance between two distributions is determined with the MMD which gives us

\begin{equation} \label{ambiguity set}
\mathcal{P} =  \left\{ P : \text{MMD} (P, P_N) \leq \varepsilon \right\}.
\end{equation}

To radius $\varepsilon$ is chosen through constructing a bootstrap MMD ambiguity set as described in \cite{Yassine_22} and can be seen in Algorithm \ref{alg:Bootstrap}. This procedure requires a number of bootstrap samples $B$ to be chosen, as well as confidence level $\beta$. It then utilizes kernels $k(\boldsymbol{\delta}^{[i]}, \boldsymbol{\delta}^{[j]}) \in \mathbb{R}$ to define the (biased) MMD estimator as 

\begin{equation} \label{ambiguity set approx}
\widehat{\text{MMD}} (\tilde{P}, P_N) = \sum_{i,j = 1}^N k(\boldsymbol{\delta}^{[i]}, \boldsymbol{\delta}^{[j]}) + k(\tilde{\boldsymbol{\delta}}^{[i]}, \tilde{\boldsymbol{\delta}}^{[j]}) - 2 k(\boldsymbol{\delta}^{[i]}, \tilde{\boldsymbol{\delta}}^{[j]})
\end{equation}

with $\tilde{\boldsymbol{\delta}}^{[n]}, n = 1,...,N$, denoting a bootstrap sample of $P_N$ where samples are drawn with replacement from $\boldsymbol{\delta}^{[1:N]}$. Finally, $\widehat{\text{MMD}} (\tilde{P}, P_N)$ is calculated for all $B$ bootstrap samples, the results are saved in list and $\varepsilon$ is chosen as the $\textit{ceil}(B \beta)$-th element of the sorted list. 


\begin{algorithm}
	\caption{Bootstrap MMD ambiguity set}
	\label{alg:Bootstrap}
	\hspace*{\algorithmicindent} \textbf{Input}: Scenarios $ \boldsymbol{\delta}^{[1:N]} $, Number of bootstrap samples $B$, Confidence level $\beta$ \\
	\hspace*{\algorithmicindent} \textbf{Output}: Gram matrix $\boldsymbol{K}$, Radius of MMD ambiguity set $\varepsilon$
	\begin{algorithmic}[1]
		\State $\boldsymbol{K} \gets \textit{kernel}(\boldsymbol{\delta}, \boldsymbol{\delta})$
		\For{$m = 1, \dots , B$}
			\State $I \gets N$ numbers from $\{1, \dots N \}$ with replacement
			\State $K_x \gets \sum_{i,j = 1}^N K_{ij};$
			\State $K_y \gets \sum_{i,j \in I} K_{ij};$
			\State $K_{xy} \gets \sum_{j \in I} \sum_{i = 1}^N K_{ij}$;
			\State MMD$[m] \gets \frac{1}{N^2} \left( K_x + K_y - 2 K_{xy} \right) ;$
		\EndFor
		\State MMD $\gets$ \textit{sort}(MMD)
		\State $\varepsilon \gets$ MMD$\left[ \textit{ceil} (B \beta) \right]$
	\end{algorithmic}
\end{algorithm}

\subsection{Constraint Reformulation}

With a given MMD ambiguity set $\mathcal{P}$, the feasible set of each constraint \ref{risk constraints} is given as

\begin{equation} \label{feasible set}
	Z_i :=  \left\{ \boldsymbol{u}_{0:H} \in \mathcal{U}^{H+1} : \inf\limits_{P \in \mathcal{P}}P \left[ \tilde{h}_i(\boldsymbol{u}_{0:H},  \boldsymbol{\delta}) \leq 0 \right] \geq 1 - \alpha \right\}.
\end{equation}

with $\tilde{h}_i(\boldsymbol{u}_{0:H},  \boldsymbol{\delta} =  h_i(\boldsymbol{u}_{0:H},  \boldsymbol{x}_{0:H},  \boldsymbol{y}_{0:H})$

We can now use the ambiguity set $\mathcal{P}$ defined in Sec. \ref{SubSec:MMD} with the radius $\varepsilon$, as well as the kernel matrix $\boldsymbol{K}$, that has been obtained in the process, to reformulate the feasible set. By following the steps described in \cite{Yassine_22}, we can obtain the new reformulated feasible set as

\begin{subequations}
  \begin{empheq}[right = \empheqrbrace, left= Z_i \coloneqq \empheqlbrace \boldsymbol{u}_{0:H} \in \mathcal{U}^{H+1} :]{align}
    & g_0 + \frac{1}{N}\sum_{n = 1}^N (\boldsymbol{K}\boldsymbol{\gamma})_n + \varepsilon \sqrt{\boldsymbol{\gamma}^\text{T}\boldsymbol{K}\boldsymbol{\gamma}} \leq t \alpha \\
    & [\tilde{h}_i(\boldsymbol{u}_{0:H},  \boldsymbol{\delta}^{[n]}) + t]_+ \leq g_0 + (\boldsymbol{K}\boldsymbol{\gamma})_n, \; n = 0,...,N \\
    & g_0 \in \mathbb{R}, \boldsymbol{\gamma} \in \mathbb{R}^N, t \in \mathbb{R}
  \end{empheq}
\end{subequations}

where $[\cdot]_+ = \text{max}(0, \cdot)$ denotes the max operator. The new set also contains new variables that have to be taken into account when solving the OCP. The variables $g_0$, $\gamma$ and $t$ are the degrees of freedom that $u$ has at each constraint. The former are parameters of the RKHS function that was used to transform the problem into a kernel machine learning problem while $t$ has been introduced as a way to relax the risk factor $\alpha$.

This approximation can then be repeated for all other constraints $\boldsymbol{h(\cdot)}$ to obtrain the feasible sets $Z_i, i = 1,...,n_c.$ We can then use those constraints to formulate the OCP as

\begin{subequations}
\begin{align}
\begin{split}
\min\limits_{\boldsymbol{u}_{0:H}, \overline{J_H}}  &\overline{J_H}
\end{split}\\
\begin{split}
\text{s.t.}\; &\forall n = 1,...,N, \;  \forall t = 0,1,...,H
\end{split}\\
\begin{split}\label{systemc1}
&\boldsymbol{x}_{t+1}^{[n]} = \boldsymbol{f}_{\boldsymbol{\theta}^{[n]}} \left( \boldsymbol{x}_{t}^{[n]} , \boldsymbol{u}_t \right) + \boldsymbol{v}_{t}^{[n]}
\end{split}\\
\begin{split}\label{systemc2}
&\boldsymbol{y}_{t} = \boldsymbol{g}_{\boldsymbol{\theta}^{[n]}} \left( \boldsymbol{x}_{t}^{[n]}, \boldsymbol{u}_t \right) + \boldsymbol{w}_{t}^{[n]}
\end{split}\\
\begin{split}
&J_H(\boldsymbol{u}_{0:H},  \boldsymbol{x}_{0:H}^{[n]},  \boldsymbol{y}_{0:H}^{[n]})  \leq \overline{J_H}
\end{split}\\
\begin{split}
 &\boldsymbol{u}_{0:H} \in Z_i, \forall i = 1, ...,n_c
\end{split}
\end{align}
\label{OCP}
\end{subequations}

As described in Sec. \ref{Problem Formulation}, we are minimizing a cost function. To ensure a robust solution, we are using the worst-case cost over all the scenarios, i.e. the cost $J_H^{[n]}$ must be lower lower or equal $\overline{J_H}$ for all scenarios $n = 1,...,N$. The system dynamics are included through the constraints \ref{systemc1} and \ref{systemc2} and must be fulfilled for all scenarios as well. Lastly, the input $u_{0:H}$ is restricted to the feasible sets $Z_i$, i.e. $u_{0:H}$ must be an element of all $Z_i, i = 1,...,n_c$.

The optimization problem \ref{OCP} is deterministic and, with the uncertainties removed, can be solved with well known methods. 




