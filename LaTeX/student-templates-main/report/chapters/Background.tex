\chapter{Technical Background} \label{Technical Background}

In this chapter, an approach is outlined that allows us to effectively draw samples from the unknown distribution defined in Section \ref{Problem Statement} using PMCMC methods. In Section \ref{PGibbs sampling}, we explain how to draw samples from the posterior distribution of model parameters and state trajectories and how to use them to generate scenarios. An existing method that uses the scenarios to formulate an optimal control problem to find a robust solution is then described in Section \ref{Scenario Approach}, and the disadvantage of this approach is highlighted.

\section{Particle Markov Chain Monte Carlo Methods} \label{PGibbs sampling}

For practical applications, the known priors $p(\boldsymbol{\theta})$ and $p(\boldsymbol{x}_{\text{-}T})$ and the observations $\mathbb{D}$ must be used to infer the posterior $p(\boldsymbol{\theta}, \boldsymbol{x}_{\text{-}T:\text{-}1}\mid \mathbb{D})$. While this is not the true distribution $p(\boldsymbol{\theta}, \boldsymbol{x}_{\text{-}T:\text{-}1})$, it is necessary since the repeated propagation of $p(\boldsymbol{x}_{\text{-}T})$ would otherwise cause an excessively large variance in $p(\boldsymbol{x}_{\text{-}1})$ making stochastic OCP infeasible. We utilize PMCMC methods to draw samples from the posterior distribution. These methods were introduced in \cite{Andrieu_10} and will be summarized in this section.

We use Particle Gibbs (PG) to bypass the issue of an analytically intractable $p(\boldsymbol{\theta}, \boldsymbol{x}_{\text{-}T:\text{-}1}\mid \mathbb{D})$ by iteratively drawing samples from $p(\boldsymbol{\theta} \mid \boldsymbol{x}_{\text{-}T:\text{-}1}, \mathbb{D})$ and $p(\boldsymbol{x}_{\text{-}T:\text{-}1}\mid \boldsymbol{\theta}, \mathbb{D})$. We continually update the distributions with the previously drawn set, i.e., $\boldsymbol{x}_{\text{-}T:\text{-}1}^{[n]}$ is drawn from $p(\boldsymbol{x}_{\text{-}T:\text{-}1}^{[n]}\mid \boldsymbol{\theta}^{[n]}, \mathbb{D})$ and $\boldsymbol{\theta}^{[n+1]}$ is then drawn from $p(\boldsymbol{\theta}^{[n+1]}\mid \boldsymbol{x}_{\text{-}T:\text{-}1}^{[n]}, \mathbb{D})$. This is repeated until the desired number of samples has been achieved.

To ensure that the samples drawn through this method are an accurate representation of the distribution  $p(\boldsymbol{\theta}, \boldsymbol{x}_{\text{-}T:\text{-}1})$, additional steps are taken. For one, the first $N_p$ samples must be discarded as they are heavily reliant on the initialization and might show a strong bias. This burn-in period should be chosen large enough that this bias is no longer reflected in the samples. The samples should also be independent of each other which is not given with this method as each $\boldsymbol{\theta}^{[n]}$ is dependent on $\boldsymbol{x}_{\text{-}T:\text{-}1}^{[n]}$ which in turn is dependent on $\boldsymbol{\theta}^{[n-1]}$. As such, measures must be taken to reduce the correlation between samples as much as possible. One approach to do this is thinning, where only every $n_d$-th sample is used, and the other samples are discarded. By increasing this parameter, the samples become more uncorrelated, but there will also be a larger amount of samples created, which leads to inefficiency.

\begin{algorithm}[t]
	\caption{Scenario generation}\label{alg:PGibbs}
	\hspace*{\algorithmicindent} \textbf{Input}: Dataset $\mathbb{D}$, parametric model $\{\boldsymbol{f}_{\boldsymbol{\theta}}(\cdot), \boldsymbol{g}_{\boldsymbol{\theta}}(\cdot), \boldsymbol{\mathcal{V}}_{\boldsymbol{\theta}}, \boldsymbol{\mathcal{W}}_{\boldsymbol{\theta}}\}$, \\
	\hspace*{\algorithmicindent} \hspace*{\algorithmicindent} priors $p(\boldsymbol{\theta})$ and $p(\boldsymbol{x}_{\text{-}T})$, $N, H, T$ \\
	\hspace*{\algorithmicindent} \textbf{Output}: Scenarios $ \boldsymbol{\delta}^{[1:N]} = \{ \boldsymbol{\theta}, \boldsymbol{x}_0, \boldsymbol{v}_{0:H}, \boldsymbol{w}_{0:H}\}^{[1:N]}$
	\begin{algorithmic}[1]
		\For{$n = 1, \dots , N$}
			\State Sample $\{ \boldsymbol{\theta}, \boldsymbol{x}_{\text{-}T:\text{-}1} \}^{[n]}$ from $p\left( \boldsymbol{\theta}, \boldsymbol{x}_{\text{-}T:\text{-}1} \mid \mathbb{D} \right)$ using a PG sampler
			\For{$t = \text{-}1, \dots , H$}
				\State Sample $\boldsymbol{v}_t^{[n]}$ from $\boldsymbol{\mathcal{V}}_{\boldsymbol{\theta}^{[n]}}$
				\State Sample $\boldsymbol{w}_t^{[n]}$ from $\boldsymbol{\mathcal{W}}_{\boldsymbol{\theta}^{[n]}}$
			\EndFor
			\State $\boldsymbol{x}_0^{[n]} \gets \boldsymbol{f}_{\boldsymbol{\theta}^{[n]}} \left( \boldsymbol{x}_{\text{-} 1}^{[n]}, \boldsymbol{u}_{\text{-} 1} \right) + \boldsymbol{v}_{\text{-} 1}^{[n]}$
		\EndFor
	\end{algorithmic}
\end{algorithm}

The samples $\{\boldsymbol{\theta}, \boldsymbol{x}_{\text{-}T:\text{-}1}\}^{[1:N]}$ can be used to generate so-called scenarios $\boldsymbol{\delta}^{[1:N]}$, which are samples from the distribution $p(\boldsymbol{\theta}, \boldsymbol{x}_0, \boldsymbol{v}_{0:H}, \boldsymbol{w}_{0:H} \mid \mathbb{D})$ and represent possible future system behavior depending on $\boldsymbol{u}_{0:H}$. The generation of these scenarios is outlined in Algorithm \ref{alg:PGibbs}. The parameters $\boldsymbol{\theta}^{[n]}$ are obtained via PMCMC, and through it we also know the system dynamics and noise distributions which can be used to draw samples of both the processing noise $\boldsymbol{v}_{0:H}$ and measurement noise $\boldsymbol{w}_{0:H}$, which can be seen in the lines 4 and 5 of the Algorithm. Those samples can then be combined with the $\boldsymbol{x}_{\text{-}T:\text{-}1}$, or more precisely $\boldsymbol{x}_{\text{-}1}$ to find the initial state $\boldsymbol{x}_{0}$ to complete the scenario $\boldsymbol{\delta} = \{ \boldsymbol{\theta}, \boldsymbol{x}_0, \boldsymbol{v}_{0:H}, \boldsymbol{w}_{0:H}\}$. How these scenarios can be used to find an optimal input $\boldsymbol{u}_{0:H}$ is described in the next section.


\section{Scenario Approach} \label{Scenario Approach}

In the previous section, we have generated the scenarios $ \boldsymbol{\delta}^{[1:N]} = \{ \boldsymbol{\theta}, \boldsymbol{x}_0, \boldsymbol{v}_{0:H}, \boldsymbol{w}_{0:H}\}^{[1:N]}$. Each scenario can be used to describe a possible future trajectory based on the input $\boldsymbol{u}_{0:H}$. As such, they can be used to reformulate the chance constraints to ensure a robust solution. We do this by ensuring that the constraints $\boldsymbol{h}(\cdot)$ are satisfied for every scenario, i.e., every possible known future satisfies the constraints \cite{Garatti_22}. As such, the constraints can be written as

\begin{equation}
	 \text{max}(\boldsymbol{h}(\boldsymbol{u}_{0:H},  \boldsymbol{x}_{0:H}^{[n]},  \boldsymbol{y}_{0:H}^{[n]})) \leq 0, \; \forall n = 1, ..., N .
\end{equation}

This approach is effective at finding a robust solution. However, the constraints no longer contain the risk factor $\alpha$, which means that determining performance guarantees requires further calculations. This is a computationally complex process where a lot of subsets of the scenarios $\boldsymbol{\delta}^{[1:N]}$ have to be tested through a greedy algorithm \cite{Garatti_21} to find the smallest possible subsample. The cardinality of this sub-sample can then be used to give a guarantee. On top of that, it is also impossible to control the risk factor without decreasing or increasing the number of scenarios used in the optimization. For certain applications, the problem might become infeasible. As such, there is a need for an alternative approach. 